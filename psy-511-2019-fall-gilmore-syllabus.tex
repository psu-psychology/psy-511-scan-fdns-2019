\documentclass[]{article}
\usepackage{lmodern}
\usepackage{amssymb,amsmath}
\usepackage{ifxetex,ifluatex}
\usepackage{fixltx2e} % provides \textsubscript
\ifnum 0\ifxetex 1\fi\ifluatex 1\fi=0 % if pdftex
  \usepackage[T1]{fontenc}
  \usepackage[utf8]{inputenc}
\else % if luatex or xelatex
  \ifxetex
    \usepackage{mathspec}
  \else
    \usepackage{fontspec}
  \fi
  \defaultfontfeatures{Ligatures=TeX,Scale=MatchLowercase}
\fi
% use upquote if available, for straight quotes in verbatim environments
\IfFileExists{upquote.sty}{\usepackage{upquote}}{}
% use microtype if available
\IfFileExists{microtype.sty}{%
\usepackage{microtype}
\UseMicrotypeSet[protrusion]{basicmath} % disable protrusion for tt fonts
}{}
\usepackage[margin=1in]{geometry}
\usepackage{hyperref}
\hypersetup{unicode=true,
            pdftitle={PSY 511 Fall 2019 Syllabus},
            pdfborder={0 0 0},
            breaklinks=true}
\urlstyle{same}  % don't use monospace font for urls
\usepackage{longtable,booktabs}
\usepackage{graphicx,grffile}
\makeatletter
\def\maxwidth{\ifdim\Gin@nat@width>\linewidth\linewidth\else\Gin@nat@width\fi}
\def\maxheight{\ifdim\Gin@nat@height>\textheight\textheight\else\Gin@nat@height\fi}
\makeatother
% Scale images if necessary, so that they will not overflow the page
% margins by default, and it is still possible to overwrite the defaults
% using explicit options in \includegraphics[width, height, ...]{}
\setkeys{Gin}{width=\maxwidth,height=\maxheight,keepaspectratio}
\IfFileExists{parskip.sty}{%
\usepackage{parskip}
}{% else
\setlength{\parindent}{0pt}
\setlength{\parskip}{6pt plus 2pt minus 1pt}
}
\setlength{\emergencystretch}{3em}  % prevent overfull lines
\providecommand{\tightlist}{%
  \setlength{\itemsep}{0pt}\setlength{\parskip}{0pt}}
\setcounter{secnumdepth}{0}
% Redefines (sub)paragraphs to behave more like sections
\ifx\paragraph\undefined\else
\let\oldparagraph\paragraph
\renewcommand{\paragraph}[1]{\oldparagraph{#1}\mbox{}}
\fi
\ifx\subparagraph\undefined\else
\let\oldsubparagraph\subparagraph
\renewcommand{\subparagraph}[1]{\oldsubparagraph{#1}\mbox{}}
\fi

%%% Use protect on footnotes to avoid problems with footnotes in titles
\let\rmarkdownfootnote\footnote%
\def\footnote{\protect\rmarkdownfootnote}

%%% Change title format to be more compact
\usepackage{titling}

% Create subtitle command for use in maketitle
\providecommand{\subtitle}[1]{
  \posttitle{
    \begin{center}\large#1\end{center}
    }
}

\setlength{\droptitle}{-2em}

  \title{PSY 511 Fall 2019 Syllabus}
    \pretitle{\vspace{\droptitle}\centering\huge}
  \posttitle{\par}
    \author{}
    \preauthor{}\postauthor{}
    \date{}
    \predate{}\postdate{}
  

\begin{document}
\maketitle

\hypertarget{foundations-of-cognitive-and-affective-neuroscience}{%
\section{Foundations of Cognitive and Affective
Neuroscience}\label{foundations-of-cognitive-and-affective-neuroscience}}

\includegraphics[height=200px]{https://allkindsofminds.files.wordpress.com/2013/02/brain-surgery}

\hypertarget{psy-511.001-fall-2019}{%
\subsection{PSY 511.001, Fall 2019}\label{psy-511.001-fall-2019}}

\hypertarget{instructor}{%
\subsection{Instructor}\label{instructor}}

Rick O. Gilmore, Ph.D. Professor of Psychology 114 Moore Building

+1 (814) 865-3664 rogilmore AT-SIGN psu PERIOD edu
\href{http://doodle.com/rickgilmore}{Schedule an appointment}

\url{http://www.personal.psu.edu/rog1}
\url{http://gilmore-lab.github.io} \url{http://databrary.org}

\hypertarget{meeting-location-and-time}{%
\subsection{Meeting Location and Time}\label{meeting-location-and-time}}

Wed \& Fri 2:30-3:45 pm, 444 Moore August 28 - December 13, 2019 Course
15384

\hypertarget{syllabus}{%
\subsection{Syllabus}\label{syllabus}}

You can find a PDF version of the syllabus at
\url{https://psu-psychology.github.io/psy-511-scan-fdns-2019/psy-511-2019-fall-gilmore-syllabus.pdf}.

\hypertarget{about-the-course}{%
\subsection{About the course}\label{about-the-course}}

The first scientific psychologists were physiologists fascinated by the
possibility of understanding the mind by studying the brain. In this
course, we will explore the historical roots and contemporary challenges
associated with the study of biological approaches to complex adaptive
behavior. In doing so, we will read and examine critically primary
source readings that discuss basic patterns and processes of brain
structure and function. The goal is to provide students with a basic
foundation of knowledge about the structures and functions of the
nervous system that can provide the basis for future study.

This course is one of two required courses for the
\href{http://psych.la.psu.edu/graduate/program-areas/cross-cutting-program-initiatives/specialization-in-cognitive-and-affective-neuroscience/specialization-in-cognitive-and-affective-neuroscience}{Specialization
in Cognitive and Affective Neuroscience (SCAN)}.

\hypertarget{prerequisites}{%
\subsection{Prerequisites}\label{prerequisites}}

Undergraduate coursework in neuroscience or physiological psychology
such as the equivalents of PSYCH 260 or BIO 469/470.

\hypertarget{schedule}{%
\section{Schedule}\label{schedule}}

\begin{center}\rule{0.5\linewidth}{\linethickness}\end{center}

\hypertarget{week-1}{%
\subsection{Week 1}\label{week-1}}

\hypertarget{wed-aug-28}{%
\subsubsection{Wed, Aug 28}\label{wed-aug-28}}

\textbf{NO CLASS}

\hypertarget{fri-aug-30}{%
\subsubsection{Fri, Aug 30}\label{fri-aug-30}}

\begin{itemize}
\tightlist
\item
  Topics

  \begin{itemize}
  \tightlist
  \item
    Structure of the course, Read BW\footnote{BW refers to the
      \emph{Behavioral Neuroscience} text by Breedlove and Watson.}
    1:1-21.
  \item
    Does neuroscience need behavior? Does behavioral science need the
    brain?
  \item
    Methods in neuroscience
  \end{itemize}
\item
  Readings

  \begin{itemize}
  \tightlist
  \item
    (recommended) Krakauer, J. W., Ghazanfar, A. A., Gomez-Marin, A.,
    MacIver, M. A., \& Poeppel, D. (2017). Neuroscience needs behavior:
    Correcting a reductionist bias. Neuron, 93(3), 480--490. Retrieved
    from \url{http://dx.doi.org/10.1016/j.neuron.2016.12.041}
  \item
    \url{https://en.wikibooks.org/wiki/Cognitive_Psychology_and_Cognitive_Neuroscience/Behavioural_and_Neuroscience_Methods}.
  \end{itemize}
\item
  Materials

  \begin{itemize}
  \tightlist
  \item
    Lecture notes \textbar{}
    \href{lectures/511-2019-08-30-intro-methods.html}{HTML slides}
  \item
    More on \href{https://www.youtube.com/watch?v=Ok9ILIYzmaY}{MRI
    physics}
  \end{itemize}
\end{itemize}

\begin{center}\rule{0.5\linewidth}{\linethickness}\end{center}

\hypertarget{week-2}{%
\subsection{Week 2}\label{week-2}}

\hypertarget{wed-sep-4}{%
\subsubsection{Wed, Sep 4}\label{wed-sep-4}}

\begin{itemize}
\tightlist
\item
  Topics

  \begin{itemize}
  \tightlist
  \item
    Methods in neuroscience, Read BW 2:51-57, 3:88-92.
  \end{itemize}
\item
  Materials

  \begin{itemize}
  \tightlist
  \item
    Lecture notes \textbar{} \href{}{HTML slides} \textbar{}
    \href{}{PDF}
  \item
    \url{https://en.wikibooks.org/wiki/Cognitive_Psychology_and_Cognitive_Neuroscience/Behavioural_and_Neuroscience_Methods}
  \item
    (Optional) Cohen, M. X. (2017). Where Does EEG Come From and What
    Does It Mean? \emph{Trends in Neurosciences}, \emph{40}(4),
    208--218. Retrieved from
    \url{http://dx.doi.org/10.1016/j.tins.2017.02.004}
  \item
    (Optional) Logothetis, N. K., Pauls, J., Augath, M., Trinath, T., \&
    Oeltermann, A. (2001). Neurophysiological investigation of the basis
    of the fMRI signal. \emph{Nature}, \emph{412}(6843), 150--157.
    Retrieved January 20, 2016, from
    \url{http://www.nature.com/nature/journal/v412/n6843/abs/412150a0.html}
  \item
    (Optional) Hillman, E. M. C. (2014). Coupling mechanism and
    significance of the BOLD signal: a status report. \emph{Annual
    Review of Neuroscience}, \emph{37}, 161--181. Retrieved from
    \url{http://dx.doi.org/10.1146/annurev-neuro-071013-014111}.
  \end{itemize}
\end{itemize}

\hypertarget{fri-sep-6}{%
\subsubsection{Fri, Sep 6}\label{fri-sep-6}}

\begin{itemize}
\tightlist
\item
  Topics

  \begin{itemize}
  \tightlist
  \item
    Neuroanatomy. Read BW 2:36-51.
  \end{itemize}
\item
  Materials

  \begin{itemize}
  \tightlist
  \item
    Lecture notes \textbar{} \href{}{HTML slides} \textbar{}
    \href{}{PDF}
  \end{itemize}
\end{itemize}

\begin{center}\rule{0.5\linewidth}{\linethickness}\end{center}

\hypertarget{week-3}{%
\subsection{Week 3}\label{week-3}}

\hypertarget{wed-sep-11}{%
\subsubsection{Wed, Sep 11}\label{wed-sep-11}}

\begin{itemize}
\tightlist
\item
  Topics

  \begin{itemize}
  \tightlist
  \item
    Neuroanatomy. Read BW 2:36-51.
  \end{itemize}
\item
  Materials

  \begin{itemize}
  \tightlist
  \item
    Lecture notes \textbar{} \href{}{HTML slides} \textbar{}
    \href{}{PDF}
  \end{itemize}
\end{itemize}

\hypertarget{fri-sep-13}{%
\subsubsection{Fri, Sep 13}\label{fri-sep-13}}

\begin{itemize}
\tightlist
\item
  Topics

  \begin{itemize}
  \tightlist
  \item
    Wrap-up on neuroanatomy
  \end{itemize}
\item
  Materials

  \begin{itemize}
  \tightlist
  \item
    Lecture notes \textbar{} \href{}{HTML slides} \textbar{}
    \href{}{PDF}
  \end{itemize}
\end{itemize}

\begin{center}\rule{0.5\linewidth}{\linethickness}\end{center}

\hypertarget{week-4}{%
\subsection{Week 4}\label{week-4}}

\hypertarget{wed-sep-18}{%
\subsubsection{Wed, Sep 18}\label{wed-sep-18}}

\begin{itemize}
\tightlist
\item
  Topics

  \begin{itemize}
  \tightlist
  \item
    \textbf{Neuroanatomy Lab}.
  \end{itemize}
\item
  Materials

  \begin{itemize}
  \tightlist
  \item
    Neuranatomy lab
    \href{handouts/neuroanatomy-lab-handout.pdf}{handout}
  \end{itemize}
\end{itemize}

\hypertarget{fri-sep-20}{%
\subsubsection{Fri, Sep 20}\label{fri-sep-20}}

\begin{itemize}
\tightlist
\item
  Topics

  \begin{itemize}
  \tightlist
  \item
    Cellular neuroanatomy. Read BW 2:24-35.
  \end{itemize}
\item
  Reading

  \begin{itemize}
  \tightlist
  \item
    Zeng, H., \& Sanes, J. R. (2017). Neuronal cell-type classification:
    challenges, opportunities and the path forward. \emph{Nature Reviews
    Neuroscience}. Retrieved from
    \url{http://dx.doi.org/10.1038/nrn.2017.85}.
  \item
    Oliveira, J. F., Sardinha, V. M., Guerra-Gomes, S., Araque, A., \&
    Sousa, N. (2015). Do stars govern our actions? Astrocyte involvement
    in rodent behavior. \emph{Trends in Neurosciences}, \emph{38}(9),
    535--549. Retrieved from
    \url{http://dx.doi.org/10.1016/j.tins.2015.07.006}
  \end{itemize}
\item
  Materials

  \begin{itemize}
  \tightlist
  \item
    Lecture notes \textbar{} \href{}{HTML slides} \textbar{}
    \href{}{PDF}
  \end{itemize}
\end{itemize}

\begin{center}\rule{0.5\linewidth}{\linethickness}\end{center}

\hypertarget{week-5}{%
\subsection{Week 5}\label{week-5}}

\hypertarget{wed-sep-25}{%
\subsubsection{Wed, Sep 25}\label{wed-sep-25}}

\begin{itemize}
\tightlist
\item
  Topics

  \begin{itemize}
  \tightlist
  \item
    \textbf{Quiz 1}. \textbar{} \href{}{Download} \textbar{}. Due at
    start of class on Friday, September 27, 2019.
  \item
    Neurophysiology. Read BW 3:61-78.
  \end{itemize}
\item
  Materials

  \begin{itemize}
  \tightlist
  \item
    Lecture notes \textbar{} \href{}{HTML slides} \textbar{}
    \href{}{PDF}
  \end{itemize}
\end{itemize}

\hypertarget{fri-sep-27}{%
\subsubsection{Fri, Sep 27}\label{fri-sep-27}}

\begin{itemize}
\tightlist
\item
  \textbf{Quiz 1 due}. \href{}{Submit here}.
\item
  Topics

  \begin{itemize}
  \tightlist
  \item
    Neural communication. Read BW 3:78-92.
  \item
    Neurochemistry. Read BW: 4:95-100.
  \end{itemize}
\item
  Materials

  \begin{itemize}
  \tightlist
  \item
    Lecture notes \textbar{} \href{}{HTML slides} \textbar{}
    \href{}{PDF}
  \end{itemize}
\end{itemize}

\begin{center}\rule{0.5\linewidth}{\linethickness}\end{center}

\hypertarget{week-6}{%
\subsection{Week 6}\label{week-6}}

\hypertarget{wed-oct-2}{%
\subsubsection{Wed, Oct 2}\label{wed-oct-2}}

\begin{itemize}
\tightlist
\item
  Topics

  \begin{itemize}
  \tightlist
  \item
    Neurochemistry II. Read BW 4:101-130.
  \end{itemize}
\item
  Materials

  \begin{itemize}
  \tightlist
  \item
    Lecture notes \textbar{} \href{}{HTML slides} \textbar{}
    \href{}{PDF}
  \end{itemize}
\end{itemize}

4:00 pm Mark Blumberg (University of Iowa) Neuroscience Seminar

\hypertarget{fri-oct-4}{%
\subsubsection{Fri, Oct 4}\label{fri-oct-4}}

\begin{itemize}
\tightlist
\item
  Topic

  \begin{itemize}
  \tightlist
  \item
    Hormones. 5:125-154. Read BW 5:131-159.
  \item
    Brain/gut connection
  \end{itemize}
\item
  Reading

  \begin{itemize}
  \tightlist
  \item
    Sarkar, A., Lehto, S. M., Harty, S., Dinan, T. G., Cryan, J. F., \&
    Burnet, P. W. J. (2016). Psychobiotics and the manipulation of
    bacteria-gut-brain signals. \emph{Trends in Neurosciences},
    \emph{39}(11), 763--781. Retrieved from
    \url{http://dx.doi.org/10.1016/j.tins.2016.09.002}
  \end{itemize}
\item
  Materials

  \begin{itemize}
  \tightlist
  \item
    Lecture notes \textbar{} \href{}{HTML slides} \textbar{}
    \href{}{PDF}
  \end{itemize}
\end{itemize}

\begin{center}\rule{0.5\linewidth}{\linethickness}\end{center}

\hypertarget{week-7}{%
\subsection{Week 7}\label{week-7}}

\hypertarget{wed-oct-9}{%
\subsubsection{Wed, Oct 9}\label{wed-oct-9}}

\begin{itemize}
\tightlist
\item
  Topics

  \begin{itemize}
  \tightlist
  \item
    Planning session for student symposium
  \end{itemize}
\end{itemize}

\hypertarget{fri-oct-11}{%
\subsubsection{Fri, Oct 11}\label{fri-oct-11}}

\begin{itemize}
\tightlist
\item
  Topics

  \begin{itemize}
  \tightlist
  \item
    Evolution \& Development. Read BW 6 \& 7.
  \end{itemize}
\item
  Reading

  \begin{itemize}
  \tightlist
  \item
    Optional
    \href{http://www.frontiersin.org/Journal/Abstract.aspx?s=742\&name=neuroanatomy\&ART_DOI=10.3389/fnana.2014.00015}{Hofman
    2014}.
  \item
    Rakic, P. (2009). Evolution of the neocortex: a perspective from
    developmental biology. Nature Reviews Neuroscience, 10(10),
    724--735. Retrieved October 5, 2015, from
    \url{http://www.nature.com/nrn/journal/v10/n10/abs/nrn2719.html}.
  \item
    Cao, M., Huang, H., \& He, Y. (2017). Developmental connectomics
    from infancy through early childhood. \emph{Trends in
    Neurosciences}, \emph{40}(8), 494--506. Retrieved from
    \url{http://dx.doi.org/10.1016/j.tins.2017.06.003}
  \end{itemize}
\item
  Materials

  \begin{itemize}
  \tightlist
  \item
    Lecture notes \textbar{} \href{}{HTML slides} \textbar{}
    \href{}{PDF}
  \end{itemize}
\end{itemize}

\begin{center}\rule{0.5\linewidth}{\linethickness}\end{center}

\hypertarget{week-8}{%
\subsection{Week 8}\label{week-8}}

\hypertarget{wed-oct-16}{%
\subsubsection{Wed, Oct 16}\label{wed-oct-16}}

\begin{itemize}
\tightlist
\item
  Topics

  \begin{itemize}
  \tightlist
  \item
    Brain development.
  \end{itemize}
\item
  Materials

  \begin{itemize}
  \tightlist
  \item
    Lecture notes \textbar{} \href{}{HTML slides} \textbar{}
    \href{}{PDF}
  \end{itemize}
\end{itemize}

\hypertarget{fri-oct-18}{%
\subsubsection{Fri, Oct 18}\label{fri-oct-18}}

\begin{itemize}
\tightlist
\item
  Topics

  \begin{itemize}
  \tightlist
  \item
    Perception. Read BW 8:230-241.
  \end{itemize}
\item
  Reading

  \begin{itemize}
  \tightlist
  \item
    Murray, M. M., Lewkowicz, D. J., Amedi, A., \& Wallace, M. T.
    (2016). Multisensory Processes: A Balancing Act across the Lifespan.
    \emph{Trends in Neurosciences}, \emph{39}(8), 567--579. Retrieved
    July 28, 2016, from
    \url{http://www.sciencedirect.com/science/article/pii/S0166223616300480}
  \end{itemize}
\item
  Materials

  \begin{itemize}
  \tightlist
  \item
    Lecture notes \textbar{} \href{}{HTML slides}
  \end{itemize}
\end{itemize}

\begin{center}\rule{0.5\linewidth}{\linethickness}\end{center}

\hypertarget{week-9}{%
\subsection{Week 9}\label{week-9}}

\hypertarget{wed-oct-23}{%
\subsubsection{Wed, Oct 23}\label{wed-oct-23}}

\begin{itemize}
\tightlist
\item
  Topics

  \begin{itemize}
  \tightlist
  \item
    Perception and Action. Read BW 10: 301:335, 11: 341:368.
  \end{itemize}
\item
  Reading

  \begin{itemize}
  \tightlist
  \item
    Nielsen, J. B. (2016). Human Spinal Motor Control. \emph{Annual
    Review of Neuroscience}, \emph{39}, 81--101. Retrieved from
    \url{http://dx.doi.org/10.1146/annurev-neuro-070815-013913}
  \end{itemize}
\item
  Materials

  \begin{itemize}
  \tightlist
  \item
    Lecture notes \textbar{} \href{}{HTML slides} \textbar{}
    \href{}{PDF}
  \end{itemize}
\end{itemize}

\hypertarget{fri-oct-25}{%
\subsubsection{Fri, Oct 25}\label{fri-oct-25}}

\begin{itemize}
\tightlist
\item
  Topics

  \begin{itemize}
  \tightlist
  \item
    Action II
  \end{itemize}
\item
  Reading

  \begin{itemize}
  \tightlist
  \item
    Shenoy, K. V., Sahani, M., \& Churchland, M. M. (2013). Cortical
    control of arm movements: A dynamical systems perspective.
    \emph{Annual Review of Neuroscience}, \emph{36}, 337--359. Retrieved
    from \url{http://dx.doi.org/10.1146/annurev-neuro-062111-150509}.
  \end{itemize}
\item
  Materials

  \begin{itemize}
  \tightlist
  \item
    Lecture notes \textbar{} \href{}{HTML slides} \textbar{} \href{}{PDF
    slides}
  \end{itemize}
\end{itemize}

\begin{center}\rule{0.5\linewidth}{\linethickness}\end{center}

\hypertarget{week-10}{%
\subsection{Week 10}\label{week-10}}

\hypertarget{wed-oct-30}{%
\subsubsection{Wed, Oct 30}\label{wed-oct-30}}

\begin{itemize}
\tightlist
\item
  Topics

  \begin{itemize}
  \tightlist
  \item
    \textbf{Quiz 2} distributed. \textbar{} \href{}{Download}
    \textbar{}. Due at start of class on Friday, November 1, 2019.
  \item
    Cognition \& language. Read BW 19.
  \end{itemize}
\item
  Reading

  \begin{itemize}
  \tightlist
  \item
    Hagoort, P., \& Indefrey, P. (2014). The neurobiology of language
    beyond single words. \emph{Annual Review of Neuroscience},
    \emph{37}, 347--362. Retrieved from
    \url{http://dx.doi.org/10.1146/annurev-neuro-071013-013847}.
  \end{itemize}
\item
  Materials

  \begin{itemize}
  \tightlist
  \item
    Lecture notes \textbar{} \href{}{HTML slides} \textbar{} \href{}{PDF
    slides}
  \end{itemize}
\end{itemize}

\hypertarget{fri-nov-1}{%
\subsubsection{Fri, Nov 1}\label{fri-nov-1}}

\begin{itemize}
\tightlist
\item
  Topics

  \begin{itemize}
  \tightlist
  \item
    Learning \& memory. Read BW 17.
  \end{itemize}
\item
  Reading

  \begin{itemize}
  \tightlist
  \item
    Squire, L. R., \& Wixted, J. T. (2011). The cognitive neuroscience
    of human memory since H.M. \emph{Annual Review of Neuroscience},
    \emph{34}, 259--288. Retrieved from
    \url{http://dx.doi.org/10.1146/annurev-neuro-061010-113720}.
  \end{itemize}
\item
  Materials

  \begin{itemize}
  \tightlist
  \item
    Lecture notes \textbar{} \href{}{HTML slides} \textbar{} \href{}{PDF
    slides}
  \end{itemize}
\end{itemize}

\begin{center}\rule{0.5\linewidth}{\linethickness}\end{center}

\hypertarget{week-11}{%
\subsection{Week 11}\label{week-11}}

\hypertarget{wed-nov-6}{%
\subsubsection{Wed, Nov 6}\label{wed-nov-6}}

\begin{itemize}
\tightlist
\item
  Topic

  \begin{itemize}
  \tightlist
  \item
    Emotion. Read BW 15.
  \end{itemize}
\item
  Materials

  \begin{itemize}
  \tightlist
  \item
    Lecture notes \textbar{} \href{}{HTML slides} \textbar{} \href{}{PDF
    slides}
  \end{itemize}
\item
  Readings + Pellman, B. A., \& Kim, J. J. (2016). What Can
  Ethobehavioral Studies Tell Us about the Brain's Fear System?
  \emph{Trends in Neurosciences}, \emph{39}(6), 420--431. Retrieved from
  \url{http://dx.doi.org/10.1016/j.tins.2016.04.001} + Hu, H. (2016).
  Reward and Aversion. \emph{Annual Review of Neuroscience}, \emph{39},
  297--324. Retrieved from
  \url{http://dx.doi.org/10.1146/annurev-neuro-070815-014106}
\end{itemize}

\hypertarget{fri-nov-8}{%
\subsubsection{Fri, Nov 8}\label{fri-nov-8}}

\begin{itemize}
\tightlist
\item
  Topics

  \begin{itemize}
  \tightlist
  \item
    Fear, stress, \& reward. Read BW 15.
  \end{itemize}
\item
  Materials

  \begin{itemize}
  \tightlist
  \item
    Lecture notes \textbar{} \href{}{HTML slides} \textbar{} \href{}{PDF
    slides}
  \end{itemize}
\item
  Readings

  \begin{itemize}
  \tightlist
  \item
    Musazzi, L., Tornese, P., Sala, N., \& Popoli, M. (2017). Acute or
    Chronic? A Stressful Question. \emph{Trends in Neurosciences}.
    Retrieved from \url{http://dx.doi.org/10.1016/j.tins.2017.07.002}
  \item
    Watabe-Uchida, M., Eshel, N., \& Uchida, N. (2017). Neural Circuitry
    of Reward Prediction Error. \emph{Annual Review of Neuroscience},
    \emph{40}, 373--394. Retrieved from
    \url{http://dx.doi.org/10.1146/annurev-neuro-072116-031109}
  \end{itemize}
\end{itemize}

\begin{center}\rule{0.5\linewidth}{\linethickness}\end{center}

\hypertarget{week-12}{%
\subsection{Week 12}\label{week-12}}

\hypertarget{wed-nov-13}{%
\subsubsection{Wed, Nov 13}\label{wed-nov-13}}

\begin{itemize}
\tightlist
\item
  Topics

  \begin{itemize}
  \tightlist
  \item
    Disorder and Disease. Read BW 16.
  \end{itemize}
\item
  Reading

  \begin{itemize}
  \tightlist
  \item
    Hunt, M. J., Kopell, N. J., Traub, R. D., \& Whittington, M. A.
    (2017). Aberrant Network Activity in Schizophrenia. \emph{Trends in
    Neurosciences}, \emph{40}(6), 371--382. Retrieved from
    \url{http://dx.doi.org/10.1016/j.tins.2017.04.003}
  \end{itemize}
\item
  Materials

  \begin{itemize}
  \tightlist
  \item
    Lecture notes \textbar{} \href{}{HTML slides} \textbar{}
    \href{}{PDF}
  \end{itemize}
\end{itemize}

\hypertarget{fri-nov-15}{%
\subsubsection{Fri, Nov 15}\label{fri-nov-15}}

\begin{itemize}
\tightlist
\item
  Topics

  \begin{itemize}
  \tightlist
  \item
    Disorder and Disease. Read BW 16.
  \end{itemize}
\item
  Reading

  \begin{itemize}
  \tightlist
  \item
    Pawluski, J. L., Lonstein, J. S., \& Fleming, A. S. (2017). The
    neurobiology of postpartum anxiety and depression. \emph{Trends in
    Neurosciences}, \emph{40}(2), 106--120. Retrieved from
    \url{http://dx.doi.org/10.1016/j.tins.2016.11.009}
  \item
    Namkung, H., Kim, S.-H., \& Sawa, A. (2017). The insula: An
    underestimated brain area in clinical neuroscience, psychiatry, and
    neurology. \emph{Trends in Neurosciences}, \emph{40}(4), 200--207.
    Retrieved from \url{http://dx.doi.org/10.1016/j.tins.2017.02.002}
  \item
    Volk, L., Chiu, S.-L., Sharma, K., \& Huganir, R. L. (2015).
    Glutamate synapses in human cognitive disorders. \emph{Annual Review
    of Neuroscience}, \emph{38}, 127--149. Retrieved from
    \url{http://dx.doi.org/10.1146/annurev-neuro-071714-033821}
  \end{itemize}
\item
  Materials

  \begin{itemize}
  \tightlist
  \item
    Lecture notes \textbar{} \href{}{HTML slides} \textbar{}
    \href{}{PDF}
  \end{itemize}
\end{itemize}

\begin{center}\rule{0.5\linewidth}{\linethickness}\end{center}

\hypertarget{week-13}{%
\subsection{Week 13}\label{week-13}}

\hypertarget{wed-nov-20}{%
\subsubsection{Wed, Nov 20}\label{wed-nov-20}}

\begin{itemize}
\tightlist
\item
  Topics

  \begin{itemize}
  \tightlist
  \item
    Networks all the way down
  \item
    \textbf{Quiz 3}. \textbar{} \href{}{Download} \textbar{}. Due at
    start of class on \textbf{Friday, November 22, 2019}.
  \end{itemize}
\item
  Supplemental Materials

  \begin{itemize}
  \tightlist
  \item
    Swanson, L. W., \& Lichtman, J. W. (2016). From Cajal to Connectome
    and Beyond. \emph{Annual Review of Neuroscience}, \emph{39},
    197--216. Retrieved from
    \url{http://dx.doi.org/10.1146/annurev-neuro-071714-033954}
  \item
    Raichle, M. E. (2015). The brain's default mode network.
    \emph{Annual Review of Neuroscience}, \emph{38}, 433--447. Retrieved
    from \url{http://dx.doi.org/10.1146/annurev-neuro-071013-014030}.
  \end{itemize}
\item
  Materials

  \begin{itemize}
  \tightlist
  \item
    Lecture notes \textbar{} \href{}{HTML slides} \textbar{}
    \href{}{PDF}
  \end{itemize}
\end{itemize}

\hypertarget{fri-nov-22}{%
\subsubsection{Fri, Nov 22}\label{fri-nov-22}}

\begin{itemize}
\tightlist
\item
  Topics

  \begin{itemize}
  \tightlist
  \item
    Reproducibility in neuroscience
  \end{itemize}
\item
  Readings

  \begin{itemize}
  \tightlist
  \item
    Gilmore, R. O., Diaz, M. T., Wyble, B. A., \& Yarkoni, T. (2017).
    Progress toward openness, transparency, and reproducibility in
    cognitive neuroscience. \emph{Annals of the New York Academy of
    Sciences}. Retrieved from \url{http://dx.doi.org/10.1111/nyas.13325}
  \item
    Gorgolewski, K. J., \& Poldrack, R. A. (2016). A practical guide for
    improving transparency and reproducibility in neuroimaging research.
    \emph{PLoS Biology}, \emph{14}(7), e1002506. Retrieved October 2,
    2016, from
    \url{http://journals.plos.org/plosbiology/article?id=10.1371/journal.pbio.1002506}
  \end{itemize}
\item
  Materials

  \begin{itemize}
  \tightlist
  \item
    Lecture notes \textbar{} \href{}{HTML slides}
  \end{itemize}
\end{itemize}

\begin{center}\rule{0.5\linewidth}{\linethickness}\end{center}

\hypertarget{thanksgiving-break-november-19---23-2018}{%
\subsection{Thanksgiving Break, November 19 - 23,
2018}\label{thanksgiving-break-november-19---23-2018}}

\begin{center}\rule{0.5\linewidth}{\linethickness}\end{center}

\hypertarget{week-14}{%
\subsection{Week 14}\label{week-14}}

\hypertarget{wed-nov-27}{%
\subsubsection{Wed, Nov 27}\label{wed-nov-27}}

\begin{itemize}
\tightlist
\item
  Topics

  \begin{itemize}
  \tightlist
  \item
    Prep for student symposium
  \end{itemize}
\end{itemize}

\hypertarget{fri-nov-29}{%
\subsubsection{Fri, Nov 29}\label{fri-nov-29}}

\begin{itemize}
\tightlist
\item
  Topics

  \begin{itemize}
  \tightlist
  \item
    Prep for student symposium
  \end{itemize}
\end{itemize}

\begin{center}\rule{0.5\linewidth}{\linethickness}\end{center}

\hypertarget{week-15}{%
\subsection{Week 15}\label{week-15}}

\hypertarget{wed-dec-3}{%
\subsubsection{Wed, Dec 3}\label{wed-dec-3}}

\begin{itemize}
\tightlist
\item
  Topics

  \begin{itemize}
  \tightlist
  \item
    Prep for student symposium
  \end{itemize}
\end{itemize}

\hypertarget{fri-dec-5}{%
\subsubsection{Fri, Dec 5}\label{fri-dec-5}}

\begin{itemize}
\tightlist
\item
  Topics

  \begin{itemize}
  \tightlist
  \item
    Student symposium
  \end{itemize}
\end{itemize}

\begin{center}\rule{0.5\linewidth}{\linethickness}\end{center}

\hypertarget{week-16}{%
\subsection{Week 16}\label{week-16}}

\hypertarget{wed-dec-11}{%
\subsubsection{Wed, Dec 11}\label{wed-dec-11}}

\begin{itemize}
\tightlist
\item
  Topics

  \begin{itemize}
  \tightlist
  \item
    Student symposium
  \end{itemize}
\end{itemize}

\hypertarget{wed-dec-13}{%
\subsubsection{Wed, Dec 13}\label{wed-dec-13}}

\begin{itemize}
\tightlist
\item
  Topics

  \begin{itemize}
  \tightlist
  \item
    Frontiers in cognitive and affective neuroscience
  \end{itemize}
\end{itemize}

\begin{center}\rule{0.5\linewidth}{\linethickness}\end{center}

\hypertarget{week-17}{%
\subsection{Week 17}\label{week-17}}

\hypertarget{wed-dec-18}{%
\subsubsection{Wed, Dec 18}\label{wed-dec-18}}

\begin{itemize}
\tightlist
\item
  Symposium \href{evaluation.html}{write-up/review papers} due by
  \textbf{noon}.
\end{itemize}

\hypertarget{evaluation}{%
\section{Evaluation}\label{evaluation}}

PSY 511 course performance will be evaluated based on the following
scheme:

\begin{longtable}[]{@{}lll@{}}
\toprule
Component & Points & \% of Grade\tabularnewline
\midrule
\endhead
Quizzes & 10 pts * 3 quizzes = 30 & 30\tabularnewline
Symposium presentation & 40 pts & 40\tabularnewline
Paper & 30 pts & 30\tabularnewline
\textbf{TOTAL} & \textbf{100} & \textbf{100}\tabularnewline
\bottomrule
\end{longtable}

\hypertarget{grading-scheme}{%
\subsubsection{Grading Scheme}\label{grading-scheme}}

\begin{longtable}[]{@{}lll@{}}
\toprule
Points & Percent & Grade\tabularnewline
\midrule
\endhead
100+ & 100+ & A+\tabularnewline
94-100 & 94-99 & A\tabularnewline
90-93 & 90-93 & A-\tabularnewline
87-89 & 87-89 & B+\tabularnewline
84-86 & 84-86 & B\tabularnewline
80-83 & 80-83 & B-\tabularnewline
77-79 & 77-79 & C+\tabularnewline
70-76 & 70-76 & C\tabularnewline
60-69 & 60-69 & D\tabularnewline
\textless{}59 & \textless{}59 & F\tabularnewline
\bottomrule
\end{longtable}

\hypertarget{student-symposium-presentation}{%
\subsection{Student symposium
presentation}\label{student-symposium-presentation}}

We will plan and host a student symposium with individual and group
presentations at the end of the semester.

\hypertarget{resource-write-up}{%
\subsection{Resource write-up}\label{resource-write-up}}

Please write-up a review of i) one of the references you discuss in your
symposium presentation or ii) another paper of your choosing in the
style of a \emph{Neuron} ``Preview'' or a \emph{Nature} ``Research
Highlights'' paper
(\href{https://www.nature.com/search?article_type=research-highlights\&order=date_desc}{example}).

Your review should be 2,000-2,500 words (6-10 pp in length) and is due
by \textbf{noon on Wednesday, December 18, 2019}.

\textbf{Do's}

\begin{itemize}
\tightlist
\item
  Put your last name and first name in the file name of your submitted
  paper. \texttt{gilmore-rick-psy-511-2018-final-paper.docx} works fine.
\item
  Submit your paper as a MS Word document or as a Google drive document
  that I can comment on using the track changes feature.
\item
  Include a cover page and title. Make sure to add page numbers.
\item
  Unpack and define all acronyms when you first mention them. Define or
  explain technical terms and concepts.
\item
  Include all end-of-paper citations in a format that is convenient to
  you and easy to extract from your reference manager.
\item
  Include author-date citations in the text, even if the article type
  (e.g., a newspaper or magazine) would not typically use them.
\item
  Use double-spacing.
\item
  Run spell-check on your paper before you submit. I also suggest
  reading your paper out loud as a way to catch run-on sentences,
  awkward phrasing, and odd word choices.
\end{itemize}

\hypertarget{resources}{%
\section{Resources}\label{resources}}

\hypertarget{text}{%
\subsection{Text}\label{text}}

Breedlove, S. M. \& Watson, N.V. (2018). \emph{Behavioral Neuroscience
(8th ed.)}. Sunderland, MA: Sinauer.

\hypertarget{web-sites}{%
\subsection{Web sites}\label{web-sites}}

\begin{itemize}
\tightlist
\item
  Course home page:
  \url{http://psu-psych.github.io/psy-511-scan-fdns-2019}
\item
  Interactive Human Brain Atlas:
  \url{http://www.med.harvard.edu/aanlib/cases/caseNA/pb9.htm}
\item
  Neurosynth (fMRI meta-analysis): \url{http://neurosynth.org}
\item
  \href{http://www.cell.com/neuron/brainview}{\emph{Neuron} Brainview}
\end{itemize}

\hypertarget{data-repositories}{%
\subsection{Data repositories}\label{data-repositories}}

\begin{itemize}
\tightlist
\item
  \href{http://openneuro.org}{OpenNeuro}
\item
  \href{http://openfmri.org}{OpenfMRI}
\end{itemize}


\end{document}
